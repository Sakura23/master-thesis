
\chapter*{Abstract}

Fraunhofer Portugal Research Center for Assistive Information and Communication
Solutions is currently developing a system to monitor the fill status of waste
containers. The introduction of a waste container fill status monitoring system
in the city of Porto, Portugal, gives rise to several opportunities. For
example, it allows the development of a detailed analysis of the city's waste
generation distribution and the optimization of waste collection routes.

This document describes the architecture design of the information system to
store and retrieve data regarding the containers' status. Furthermore, it
provides a description of several algorithms that can be used to obtain
efficient collection routes. This optimization problem is modeled as the
\textit{Capacitated Vehicle Routing Problem}. To address this problem, two
approaches were analyzed; the first involves solving the associated
\textit{Asymmetric Traveling Salesman Problem} --- in which vehicle capacity
constraints are ignored --- followed by clustering the resulting tour into
feasible routes. This approach is called \textit{route-first-cluster-second}.
The second approach relies on the usage of a construction heuristic by
Clarke and Wright.

Regarding the optimization of the \textit{Asymmetric Traveling Salesman Problem}
solution, this study compares several techniques: two construction heuristics
--- \textit{greedy} and \textit{repetitive nearest neighbor} --- and three
meta-heuristics --- \textit{hill climbing}, \textit{genetic algorithms} and
\textit{MAX-MIN ant system}. Additionally, \textit{MAX-MIN ant system} was
subjected to a parameter sensibility analysis.

Results show that \textit{MAX-MIN ant system} achieves more efficient routes
when the number of ants is higher, although it increases the algorithm's running
time.  When dealing with a scenario in which there is a limited time-frame, it
is recommended that a low number of ants is used. The algorithm was also shown
to be very sensitive to changes in parameter $\beta$, which indicates if an ant
should give more importance to the distance between two vertices or to the
pheromone levels in that arc. This analysis suggests that $\beta$ should be
close to $20$.

When evaluating the performance of the presented techniques applied to the
\textit{Capacitated Vehicle Routing Problem}, \textit{MAX-MIN ant system}
produced, in average, more efficient routes than the other approaches. 


\chapter*{Resumo}

O Centro de Pesquisa para Soluções de Informação e Comunicação Assistiva da
Fraunhofer Portugal está a desenvolver um sistema de monitorização do estado de
enchimento dos contentores de lixo. A introdução deste sistema na cidade do
Porto dá origem a várias oportunidades. Por exemplo, torna-se possível fazer uma
análise detalhada da distribuição da geração do lixo na cidade. Este projecto
permite, também, implementar um sistema de optimização das rotas de recolha do
lixo.

Este documento descreve o desenho da arquitectura do sistema de informação que
permitirá armazenar --- e disponibilizar --- a informação referente ao estado
dos contentores. O documento oferece também uma descrição de vários algoritmos
que podem ser utilizados para obter rotas de recolha eficientes. Este problema
de optimização pode ser modelado como um problema de planeamento de rotas de
veículos com capacidade limitada (CVRP).  Neste estudo, foram analisadas duas
abordagens para a resolução do CVRP.  A primeira começa por resolver o problema
do caixeiro viajante em grafos assimétricos (ATSP) --- ignorando as restrições
de capacidade --- e, subsequentemente, divide o circuito obtido em rotas que
respeitem as restrições de capacidade dos veículos. Esta técnica chama-se
\textit{route-first-cluster-second}. A segunda abordagem para resolver o CVRP
é baseada numa heurística construtiva, por Clarke e Wright.

Relativamente ao problema de optimização do problema do caixeiro viajante em
grafos assimétricos, foram comparadas várias técnicas: duas heurísticas
construtivas --- \textit{gulosa} e \textit{vizinho mais próximo repetitivo} ---
e três meta-heurísticas --- \textit{subir-a-colina}, \textit{algoritmos
genéticos} e um sistema de formigas chamado \textit{MAX-MIN ant system} (MMAS).
Além da comparação dos vários algoritmos entre si, foi também feita uma análise
de sensibilidade aos parâmetros do MMAS.

Os resultados mostram que o MMAS calcula rotas mais eficientes quando o número
de formigas é mais elevado, apesar de levar a um aumento no tempo de execução do algoritmo.
Quando aplicado a um cenário em que o tempo de execução disponível é limitado,
é recomendado que se utilize um número reduzido de formigas. Também foi possível
mostrar que o algoritmo é bastante sensível a variações no parâmetro $\beta$,
que decide se uma formiga deve dar mais importância à distância entre dois
vértices ou à quantidade de feromonas existente nesse arco. Esta análise mostrou
que o parâmetro $\beta$ deve tomar valores perto de $20$.

A análise do desempenho dos vários algoritmos, relativamente ao problema de
planeamento de rotas de veículos de capacidade limitada, mostrou que o sistema
de formigas obtém, em média, rotas mais eficientes. 

