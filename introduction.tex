\chapter{Introduction} \label{chap:intro}

\section*{}

This chapter introduces this work, by presenting its context and motivation.
Finally, it presents the project's objectives and the document structure.

\section{Context and motivation}
\label{section:context}

Municipal solid waste (MSW) production has been increasing in the last few
years, along with economic growth~\citep{McCarthy94}. This has led to the need
--- and subsequent development --- of efficient waste management solutions.
Waste management involves not only the collection, but also the transportation,
recycling and disposal of generated waste.

According to~\citet{Bhat1996}, municipal solid waste collection and disposal
represents up to 85\% of some cities' waste management budget. With this in
mind, route optimization is as an important field of study regarding the
improvement of MSW management processes.

Many cities around the world have studied and applied optimization techniques
to their collection scenarios, which are usually very different from each
other. In Portugal, however, there seems to be little research regarding this
subject. \citet{Passaro200397} describes the waste management scenario in
Portugal from 1996 and 2002, a period in which several improvements were made
due to changes in the legislation. In 2006, \citet{Magrinho20061477} further
report the legislation trends and present some statistics on the average MSW
generation rate. Concerning collection routing, \citet{Teixeira04} describes a
study for optimizing the collection of urban recyclable waste in the
center-littoral region of Portugal.




\section{Fill status monitoring}
\label{section:monitoring}

The Municipality of Porto, Portugal, is working together with the Fraunhofer
Portugal Research Center for Assistive Information and Communication Solutions
(FhP-AICOS) in order to implement a platform that allows the real-time
measurement of waste containers' fill status. This requires the deployment of
low cost sensors in each container and the development of a communication
system to gather information. On top of this platform, several applications
will then be possible. As an example, this will allow researchers to study
waste generation behaviors on a more detailed level. These monitoring systems
have been studied in places, such as Pudong New Area, in Shanghai,
China~\citep{Rovetta09,Vicentini09} and Sweden~\citep{Johansson06}.

Another application for this system is the optimization of waste collection
routes.




\section{Optimization of waste collection routes}
\label{section:optimization}

The problem of optimizing waste collection routes involves deciding, for
example, which streets must each garbage truck follow, which containers should
each one of them collect and how many trucks should a fleet for a given city
have.

One of the first articles regarding this subject was done in
1974~\citep{Beltrami74}, and it was applied to both New York and Washington
D.C., United States of America. Since then, other cities have tried to minimize
the costs by optimizing collection routes: Trabon, Turkey~\citep{Apaydin2007};
Barcelona, Spain~\citep{Bautista2004}; Athens, Greece~\citep{Karadimas2005};
Hanoi, Vietnam~\citep{Tung2000}; Porto Alegre, Brazil~\citep{Li2008} and many
others.

However, many of these studies do not have real-time information of the
containers' fill status. Usually, they are either based on statistical data
(surveys), or they ignore the containers' fill status and simply collect the
waste in every container.

Combining these techniques with the \textit{Fill Status Monitoring} platform
described in section~\ref{section:monitoring}, the municipality of Porto,
Portugal might reduce even further the collection costs.




\section{Objectives}
\label{section:objectives}

With the deployment of a fill status monitoring solution in the municipality of
Porto, there is the opportunity to develop an optimization framework for the
waste collection routes.

The first challenge is to devise and implement an architecture to store and
retrieve, when necessary, the information obtained from the container fill
sensors. This includes specifying the information workflow, the database schema
and formats to exchange data between modules.

The second goal is to analyze and compare different algorithms for the
optimization of waste collection routes so that an efficient itinerary can be
calculated within a time-frame of two hours, given the containers' fill status
and the collection vehicles' capacities.

\newpage
\section{Document structure}

The next chapter on this document, chapter~\ref{chap:sota}, provides an
overview of waste collection route optimization approaches. First, an informal
description of each scenario is given. Then, each one of the scenarios is
exposed as a mathematical formulation, followed by possible techniques that can
be applied to them. Chapter~\ref{chap:problem} summarizes the problem statement,
using the definitions presented by chapter~\ref{chap:sota}.

Chapter~\ref{chap:evaluation-framework} introduces the architecture proposal for
the management of waste containers' fill status information --- modules,
workflow and information interchange formats. It also presents the technologies
used to implement this system. Finally, this chapter presents the metrics and
datasets used for both algorithm validation and analysis.

Chapter~\ref{chap:algorithms} describes the algorithms used to optimize
collection routes and shows their validation results.
Chapter~\ref{chap:results} presents a comparative analysis of the chosen
algorithms' performance. Chapter~\ref{chap:conclusions} finalizes this document
by presenting the conclusions of this project, along with possible further
developments and future work.


