\chapter{State of the art} \label{chap:sota}

\section*{}

This chapter describes several studies regarding the optimization of solid
waste collection routes. It reviews the mathematical models used in these
studies and provides alternative optimization techniques that can be applied to
each one.

In section~\ref{section:problem} the route collection problem will be detailed.
Section~\ref{section:over-rp} provides a background on the categorisation of
routing problems. As a final introductory section, \ref{section:city-graph}
exposes the basic data structure used in waste collection problems.

Sections \ref{section:residential}, \ref{section:commercial} and
\ref{section:rrvrp} describe each one of the three possible scenarios. Each
section presents a mathematical formulation of the problem, along with techniques
for solving instances of the problem.





\section{General problem description}
\label{section:problem}

As stated by \citet{Golden01}, one can divide waste collection routing problems
into three main categories. First, there is commercial collection, which
regards waste collection from businesses and organizations like malls,
factories and such. Second, the residential collection problem involves
collecting household generated waste, usually stored in containers along the
streets of a city. Comparatively, the number of containers in the commercial
problem is significantly smaller than in the residential case.

In both of these two variants, each waste collection vehicle travels to a
container, loads its contents into its hopper and moves on to the next
container. As soon as the hopper is full, the vehicle travels to a disposal
facility (such as a landfill, or a recycling/treatment facility) and deposits
the collected waste.

The third variant --- named Rollon-Rolloff, which is described in
\cite{Bodin00} --- involves large containers, which must be transported
to the disposal facilities and replaced by empty ones. Their dimensions impose
that each vehicle can only transport a single container (either full or empty)
at a time. 

In the Rollon-Rolloff original description, each waste collection vehicle can
perform four basic operations:

\begin{itemize}
\item \textbf{Insertion trip} --- the vehicle brings an empty container from
the WDF to a new location;

\item \textbf{Removal trip} --- the vehicle picks up a full container from a
location and leaves it at the WDF;

\item \textbf{Round trip} --- the vehicle picks up a container, brings it to a
waste disposal facility (WDF) for emptying and returns it to its original
location;

\item \textbf{Exchange trip} --- the vehicle leaves the WDF with an empty
container, travels to a location with a full one and switches them, bringing it
back to the WDF.
\end{itemize}

Although round and exchange trips could be considered as being composed by
insertion and removal operations, the authors considered them as seperate
actions to avoid the need for modelling extra constraints. For example, in
round trips the same container is brought back to its original place, while
minimizing the time that location stays without a container. In the exchange
trip case, the target location may not have room for two containers, so an
extra restriction would have to be added, forcing vehicles to pick up the full
one before delivering the new, empty container.

In each of these three main variants, extra parameters can vary. For example,
consider a scenario with either a single or multiple waste disposal facilities.
In the multiple facilities case, an extra restriction may be to balance the
waste disposed at each location.







\subsection{City graph}
\label{section:city-graph}

All three variants described in section~\ref{section:problem} have a common
base for their mathematical models --- waste collection vehicles that travel
along the streets of a given city. To model a city, the common approach is to
define a graph in which each street is represented by one or two arcs
(depending if the street is one or both ways), with street intersections being
the vertices. Each arc may have several associated weights, reflecting the
street distance, the time it takes to transverse it, or some other metric. As
such, a city graph is defined as the ordered tuple $G = (V, E \cup A)$, where
$V$ represent the vertices, $E$ the (undirected) edges and $A$ the (directed)
arcs.

This definition alone does not suffice to realistically represent the possible
vehicle routes, as there are extra restrictions which must be considered,
specially regarding traffic signs. For example: at a given intersection, it may
not be possible to take a left turn if you are arriving from a specific street.
U-turns may also be forbidden, in certain intersections. 

These restrictions can be specified by defining a cost for every pair of arcs
that intersect at a given vertex. $t_{ava^\prime}$ is denoted as being the
cost to go from the arc $a$ to arc $a^\prime$ by making a turn at $v$. If the
arcs do not intersect at that vertex, or if the traffic signs forbid such a
turn, set $t_{ava^\prime} = +\infty$. Further details on this approach can
be seen in \citet{Corberan2002887}.

For now, consider the case in which there is a depot facility, located in
vertex $v_0 \in V$, and that all waste collection vehicles start and finish
their routes there. Consider that there are $K$ vehicles available, and that
each of the $K$ routes can be defined as a set of round trips, each starting at
the depot. The number of round trips in each vehicle route may or may not be
limited.

Furthermore, consider vehicles being limited to an amount $Q$ of waste that
they can carry at any given time. This can be either measured in weight or in
volume.






\section{Overview of routing problems}
\label{section:over-rp}

Before presenting the mathematical formulations that can be used to model our
three waste collecting scenarios, this provides a background on routing
problems. This is useful to understand the concepts applied in the following
sections. Routing problems are an important field in optimization, with many
different applications. Its first formulation is usually attributed to Euler's
article on the briges of Königsberg\citet{Euler1736}.

The usual underlying structure for these problems is a strongly connected mixed
graph. A mixed graph, $G = (V, E \cup A)$, contains both undirected and
directed links (edges and arcs, respectively). Being strongly connected means
that there is a directed path between every pair of vertices in $V$. With each
link, there is an associated value that represents the cost of transversing it.

Presenting the \textit{Mixed-graph Capacitated General Routing Problem}
(CGRP-m, or MCGRP), defined in \citet{Cesar02,Pandit95}. The goal is to find a
set of $K$ routes that start and end in a specific vertex $v_d \in V$, which is
called the \textit{depot}. Additionally, these routes must respect a following
set of constraints.

Consider the following subsets $E_R \subseteq E$, $A_R \subseteq A$ and $V_R
\subseteq V$. With each element of these subsets --- which shall be called
\textit{requests} --- there is an associated positive demand ($q_{a}$, $q_{e}$
or $q_{v}$). These requests must all be \textit{serviced} by one and just one
of the $K$ routes exactly once. A request being serviced means that a route, at
some point, visits it. Note that while each request may only be serviced once,
its associated element (vertex or link) may be visited multiple times;
transversing a link without servicing it is called \textit{deadheading}.
Furthermore, for each one of the $K$ routes, the sum of the serviced requests'
demands should not exceed a fixed capacity $Q$.

In order for this problem to be feasible, the service demands must fulfil two
conditions. First, each demand must not be superior to the capacity $Q$.
Second, there must be a way to partition the requests into no more than $K$
subsets, such that the total demand for each subset does not exceed $Q$.  These
two conditions can be specified by the equations \eqref{cap:individual} and
\eqref{cap:binpacking}.

\begin{align}
\forall s \in E_R \cup A_R \cup V_R : & \quad q_{s} \leq Q
	\label{cap:individual}
\end{align}
\begin{align}
\exists P : &
	\quad \bigcup P = E_R \cup A_R \cup V_R \wedge \notag{} \\
	& \quad |P| \leq K \wedge \notag{} \\
	& \quad \forall A, B \in P : A \cap B = \emptyset \wedge \notag{} \\
	& \quad \forall A \in P : \sum_{s \in A} q_s \leq Q
	\label{cap:binpacking}
\end{align}

The CGRP-m, as its name states, is a general routing problem. This means that
it is a generalization of several known and analysed routing problems. A short
list of CGRP-m specializations is now presented. In each of them, the
previously stated conditions must always be respected.

These specializations can be divided into two main categories:
\textit{capacitated arc routing problems} (CARP), with $V_R = \emptyset$, and
node routing problems, with $E_R = A_R = \emptyset$.

In the CARP category, the special case where $K = 1$ is called the
\textit{Rural Postman Problem} (RPP). In addition, if $A_R \cup E_R = E \cup
A$, the \textit{Chinese Postman Problem} (CPP) is obtained\citep{Pandit95}.
They can be classified as \textit{Directed} (DRPP and DCPP),
\textit{Undirected} (URPP and UCPP) or \textit{Mixed} (MRPP and MCPP), if $E =
\emptyset$, $A = \emptyset$ or $A \neq \emptyset \wedge E \neq \emptyset$,
respectively. DCPP and UCPP have been proven to be solvable in polynomial time
by \citet{Edmonds73}, using a matching algorithm. Mixed CPP and all RPP are
part of the NP-complete complexity
class\citep{Pandit95}.

The second category represents node routing problems. Letting $K = 1$ yields
the \textit{Steiner Graphical Travelling Salesman Problem}
(SGTSP)\citep{Letchford99}. When a SGTSP also satisfies the condition $V_R =
V$, it is called the \textit{Graphical Travelling Salesman Problem} (GTSP).
Finally, if the underlying graph is complete, the problem becomes the classical
\textit{Travelling Salesman Problem}.

On the other hand, the instances with $V_R = V$ form a widely studied variant
--- the \textit{Vehicle Routing Problem} (VRP). This problem, along with
several of its variants, is described in \citet{Toth01book}.







\section{Residential scenario}
\label{section:residential}

In the residential scenario, due to the container density per street, it is
usual to consider that vehicles should serve arcs instead of individual nodes.
Oppositely, in the commercial scenario dealt with in the next section, vehicles
will serve vertices.



\subsection{Mathematical Formulation}

Since each vehicle is also limited in its capacity, it is considered to be a
problem in the CARP category. More specifically, it can be modelled as a
\textit{Mixed Capacitated Arc Routing Problem} (MCARP).

A recent survey on CARP can be found in \citet{Wohlk08}. The problem was first
suggested and modelled, in its undirected variant, by \cite{Golden81}, with
Belenguer et al. providing a different mathematical model, as well as an
algorithm for determining lower and upper
bounds~\citep{Belenguer98,Belenguer03}.

With respect to CARP on mixed graphs (MCARP), Belenguer et
al.\citep{Belenguer06} provide a relaxed linear formulation, lower bounds for
it and several heuristics. Recently, Gouveia et al.\citep{Gouveia10} provide a
valid linear formulation for the MCARP and present benchmarks on large
datasets. This formulation will be now described.

Start by considering a graph $G^\prime = (N, A^\prime)$, where $A^\prime = A
\cup \{(i,j),(j,i) : (i,j) \in E\}$, and let $A^{\prime}_R = A_R \cup
\{(i,j),(j,i) : (i,j) \in E_R\}$ and $P = \{1, ..., K\}$. A solution is given
by the variables $(x, y, f)$, in which:

\begin{align*}
x_{ij}^p = & \quad \left\{
	\begin{array}{ll}
		1 & \quad \mbox{if vehicle $p \in O$ serves arc $(i,j) \in A^{\prime}_R$} \\
		0 & \quad \mbox{otherwise}
	\end{array}
	\right. \\
y_{ij}^p = & \quad \mbox{number of times vehicle $p \in O$ visits $(i,j) \in A^{\prime}$ without servicing it} \\
f_{ij}^p = & \quad \mbox{remaining demand serviced by vehicle $p \in P$, after transversing $(i,j) \in A^\prime$}
\end{align*}

Now, consider $d_{ij}$ and $c_{ij}$ to be the cost of transversing arc $(i,j)
\in A^\prime$, with and without servicing it, respectively. One can specify
MCARP as the following linear programming formulation:

\begin{align}
	\mbox{Minimize} & \quad
	\sum_{p \in P} \left[
		\sum_{(i,j) \in A_R^\prime} c_{ij} x^p_{ij} + 
		\sum_{(i,j) \in A_R^\prime} d_{ij} y^p_{ij}
	\right]
	\\
	\mbox{subject to} &  \notag{}
	\\
	\forall {p \in P} \quad \forall {v \in V} & \quad
		\sum_{(v,j) \in A^\prime} y^p_{vj} + \sum_{(v,j) \in A_R^\prime} x^p_{vj} =
		\sum_{(j,v) \in A^\prime} y^p_{jv} + \sum_{(j,v) \in A_R^\prime} x^p_{jv}
	\label{mcarp:continuity}
	\\
	\forall {(i,j) \in A_R} & \quad \sum_{p \in P} x^p_{ij} = 1
	\label{mcarp:arc-servicing}
	\\
	\forall {(i,j) \in E_R} & \quad \sum_{p \in P} x^p_{ij} + x^p_{ji} = 1
	\label{mcarp:edge-servicing}
	\\
	\forall {p \in P} & \quad
		\sum_{(v_0,j) \in A^\prime} y^p_{v_0j} +
		\sum_{(v_0,j) \in A_R^\prime} x^p_{v_0j}
	\leq 1
	\label{mcarp:max-vehicles}
	\\
	\forall {p \in P} \quad \forall {v \in V \setminus \{v_0\}} & \quad
		\sum_{(i,v) \in A^\prime} f^p_{iv} - \sum_{(v,i) \in A^\prime} f^p_{vi} =
		\sum_{(i,v) \in A_R^\prime} q_{iv} x^p_{iv}
	\label{mcarp:flow-1}
	\\
	\forall {p \in P} & \quad
	\sum_{(v_0,i) \in A^\prime} f^p_{v_0i} = \sum_{(i,j) \in A_R^\prime} q_{ij} x^p_{ij}
	\label{mcarp:flow-2}
	\\
	\forall {p \in P} & \quad
	\sum_{(i,v_0) \in A^\prime} f^p_{iv_0} = \sum_{(i,v_0) \in A_R^\prime} q_{iv_0} x^p_{iv_0}
	\label{mcarp:flow-3}
	\\
	\forall {p \in P} \quad \forall {(i,j) \in A^\prime} & \quad
	f^p_{ij} \leq Q(y^p_{ij} + x^p_{ij})
	\label{mcarp:capacity}
	\\
	\forall {p \in P} \quad \forall {(i,j) \in A} & \quad f^p_{ij} \geq 0 \\
	\forall {p \in P} \quad \forall {(i,j) \in A} & \quad x^p_{ij} \in \{0,1\} \\
	\forall {p \in P} \quad \forall {(i,j) \in A} & \quad y^p_{ij} \in \mathbb{Z}_{\geq0}
\end{align}

Constraints \eqref{mcarp:continuity} state that the number of vehicles that
enters a vertex must be the same as the number of vehicles that leave it.
\eqref{mcarp:arc-servicing} and \eqref{mcarp:edge-servicing} force every arc
and edge to be serviced. \eqref{mcarp:max-vehicles} ensures that each vehicle
only leaves the depot once. Flow constraints \eqref{mcarp:flow-1},
\eqref{mcarp:flow-2} and \eqref{mcarp:flow-3} force, together with the linking
constraints \eqref{mcarp:capacity}, that each vehicle performs a connected
trip. Constraints \eqref{mcarp:capacity} also handle the capacity limit for
each vehicle, when in conjugation with \eqref{mcarp:flow-2}.

Further details regarding the theory of arc routing problems, along with
possible approaches for solving them, can be found in \citet{Dror00,Assad95}.





\subsection{Solving approaches}
\label{section:mcarp-approaches}

In \citet{Belenguer06}, the authors describe three fast heuristics, adapted
from studies on the undirected version of CARP: path scanning, augment-merge
and Ulusoy's heuristic. 

\begin{description}
\item[Path scanning] \hfill \\
this heuristic builds routes sequentially. In each step, the current route is
extended (until the capacity constraints are violated) by adding the arcs that
lead to the closest arc $v \in A_R^\prime$ which needs to be serviced. In case
of ties, the heuristic may use randomly one of five criteria:
\begin{itemize}
	\item[$F_1$] maximise the cost to return to the depot;
	\item[$F_2$] minimise the cost to return to the depot;
	\item[$F_3$] maximise the ratio $q_v/c_v$;
	\item[$F_4$] minimise the ratio $q_v/c_v$;
	\item[$F_5$] if the vehicle is less than half-full, use $F_2$ and otherwise use $F_1$.
\end{itemize}

\item[Augment-merge] \hfill \\
A route is created for each required link $e \in A_R \cup E_R$, minimizing the
deadheading cost (This can easily be done using shortest path algorithms). The
next step, called \textit{Augment phase}, verifies if there are routes which
include required links in their deadheading arcs. When it happens, they are
merged together, if the capacity constraints hold.

The second phase, \textit{Merge}, takes every pair of routes $(r_0, r_1)$ and
checks if their concatenation yields a better result (while respecting the
capacity constraints), and merges them. The concatenation process is done by
finding the shortest path between the last served arc in $r_0$ to the first
served arc in $r_1$.

\item[Ulusoy's heuristic] \hfill \\
This heuristic builds a single route containing all arcs in $A_R^\prime$,
ignoring the capacity constraints, and subsequently divides it into several
feasible routes.

The first step does not need to produce an optimal route, so the authors
suggest the usage of the path-scanning heuristic and disregarding the capacity
limit. This route defines the order in which the arcs will be serviced, so let
$pos(a)$ be the arc's position on this route.

Next, the route is subdivided. This step can be done using a simple dynamic
programming technique.
\end{description}

In the same study, the authors propose new linear relaxed formulation which can
is used together with a cutting plane algorithm to determine a lower bound for
MCARP instances. 

Belenguer et al. also mention several metaheuristics applied to UCARP that
solve a majority of instances optimally\citep{Belenguer06,Belenguer03}. These
could be adapted to solve MCARP.





\section{Commercial scenario}
\label{section:commercial}

When comparing the commercial and residential collection problems, one expects
that the number of containers is significantly lower in the first case. This
allows us to model the problem as a node routing problem. Applications of this
model to the waste collection problem are described in \citet{Tung2000} and
\citet{Kim06}.

\subsection{Mathematical Formulation}

Generally, the city graph is transformed into $G^\prime = (V^\prime,
A^\prime)$, in which each vertex represents a waste container and the depot.
The arcs represent the associated cost of travelling between each pair of
vertices -- obtained through a shortest path algorithm applied to $G$. This
way, the problem becomes finding a set of routes such that each vertex is
visited exactly once. This is the \textit{Vehicle Routing Problem}, which was
described in section~\ref{section:over-rp}.

Since the graph is directed, this problem is usually named \textit{Asymmetric
Capacitated Vehicle Routing Problem} (ACVRP).  The most common mathematical
model\citep{Toth01} defines a solution as a set of binary variables $x_{ij}$,
one for each arc $(i, j)$.  Additionally, consider $r(S), S \subseteq V
\setminus \{v_0\}$ as the minimum number of vehicles to serve all vertices in
$S$. This value, that can be determined by solving the \textit{Bin Packing
Problem} (BPP), has a trivial lower bound:

\begin{equation}
\forall_{S \subseteq V^\prime \setminus \{v_0\}} \quad r(S) \geq
\lceil\frac{1}{Q} \sum_{v \in V^\prime} d_{v} \rceil
\end{equation}

These definitions allows us to define ACVRP as the following integer
programming formulation:

\begin{align}
	\mbox{Minimize} & \sum_{i \in V^\prime} \sum_{j \in V} c_{ij} x_{ij}
	\\
	\mbox{subject to} & \notag{}
	\\
	\forall_{j \in V \setminus \{v_0\}} & \quad \sum_{i \in V} x_{ij} = 1
	\label{acvrp:indegree}
	\\
	& \quad \sum_{i \in V} x_{v_0i} = K
	\label{acvrp:outdepot}
	\\
	\forall_{i \in V} & \quad \sum_{j \in V} x_{ij} = \sum_{j \in V} x_{ji}
	\label{acvrp:in-out}
	\\
	\forall_{S \subseteq V \setminus \{v_0\}, S \neq \emptyset} & \quad
	\sum_{i \notin S} \sum_{i \in S} x_ij \geq r(S)
	\label{acvrp:capacity}
	\\
	\forall_{i, j \in V} & \quad x_{ij} \in \{0, 1\}
\end{align}

Constraints \eqref{acvrp:indegree} impose that each vertex is visited exactly
once, while constraints \eqref{acvrp:outdepot} force that exactly $K$ vehicles
leave the disposal facility and \eqref{acvrp:in-out} forces that every vehicle
that visits a node must leave it. Finally, \eqref{acvrp:capacity} ensures that
each subset of vertices is serviced by at least the minimum number of vehicles
required to fulfil their demands, thus respecting the capacity constraints.




\subsection{Solving approaches}
\label{section:vrp-approaches}

In the book by Toth and Vigo\citep{Toth01book}, VRP is described with great detail.
They divide methods into the following categories:

\begin{itemize}
	\item Branch-and-bound
	\item Branch-and-cut
	\item Set-covering based algorithms
	\item Heuristics
	\item Metaheuristics
\end{itemize}

The first three categories are exact methods, based on relaxations of linear
formulations and the determination of lower bounds. Some of these approaches
have been used to solve optimally problem instances with up to 135
nodes\citep{Naddef01}. These involve the study of polytopes defined by linear
formulations and methods to apply cutting plane algorithms.

In the heuristics field, there are three subcategories\citep{Laporte01}.
\textit{Constructive heuristics} gradually build a feasible solution,
minimizing the cost in a greedy fashion. \textit{Two-phase heuristics} divide
the problem into two: clustering the vertices into routes and constructing
routes themselves. Some methods do the clustering first
(\textit{cluster-first-route-second}), while others build a single route and
consequently divide it into vehicle trips
(\textit{route-first-cluster-second})\citep{Tucker76}. The latter approach is
similar to Ulusoy's heuristic, described in
section~\ref{section:mcarp-approaches}. The third category, \textit{Improvement
methods}, is applied over the two first categories. Improvement methods can be
applied to a single route; in this case, TSP improvement techniques can be
used. Improvement methods can also several routes, exchanging vertices between
them.

The last category of methods, metaheuristics, is said to produce better results
than the previous one\citep{Gendreau01}. Some of these methods are no more than
\textit{Improvement techniques} applied on top of heuristic approaches.

As examples of metaheuristics applied to VRP, we have \textit{Simulated
Annealing}, \textit{Deterministic Annealing}, \textit{Tabu Search},
\textit{Genetic Algorithms} and \textit{Ant Systems}.  The first three
approaches start by building an initial solution (using a heuristic) and search
its neighbourhood for a better solution. The neighbourhood is usually defined
by some swap operator; for example, exchanging a subset of serviced vertices
between routes.

Genetic Algorithms starts by generating a set of initial solutions, and
proceeds to the next step by combining solutions between them, and discarding
the worst. Ant systems work by applying concepts based on the ants' pheromone
system for marking paths\citep{Colorni91}.







\section{Rollon-Rolloff scenario}
\label{section:rrvrp}

Regarding the Rollon-Rolloff case, some of the earlier studies found on the
literature are \citet{Cristallo94}, \citet{Meulemeester97}, \citet{Bodin00}.
According to Bodin et al., the first three papers assume that it is known
beforehand which trip type (see section~\ref{section:problem} must serve each
container. This problem is defined as the \textit{Rolloff-Rollon Vehicle
Routing Problem} (RRVRP).

\subsection{Mathematical Formulation}

Each given trip $t \in T$ is defined by its type and by a tuple of vertices to
visit, in a specific order. The problem of assigning trips to vehicles can be
formulated as an \textit{Asymmetric Vehicle Routing Problem} (AVRP). AVRP is
similar to the ACVRP without the vehicle capacity constraints, but differs in
that the number of vehicles is not given --- it is a value which must be
minimised. Usually, the number of vehicles must be bounded by a given interval
$[L, U]$.

In order to model the RRVRP as a AVRP, admit a new graph, $G^\prime =
(V^\prime, A^\prime)$. The vertices on this new graph are the disposal facility
node and a node for each trip to be serviced. Arc $(i, j) \in A^\prime$
represent the transition between trip $i$ and trip $j$. When $i = 0$, 
consider this transition to be the start of a route; when $j = 0$, consider
it as the end. The costs of going from the end location of a given trip to the
start location of another one are given by $c_{a}, a \in A^\prime$.

The AVRP formulation for the RRVRP can now be defined as:

\begin{align}
	\mbox{Minimize} & \quad K_A \sum_{i \in V^\prime} \sum_{j \in V} c_{ij} x_{ij} +
				K_B \sum_{j \in V} x_{v_0j}
	\\
	\mbox{subject to} & \notag{}
	\\
	\forall_{j \in V \setminus \{v_0\}} & \quad \sum_{i \in V} x_{ij} = 1
	\label{avrp:indegree}
	\\
	\forall_{i \in V} & \quad \sum_{j \in V} x_{ij} = \sum_{j \in V} x_{ji}
	\label{avrp:in-out}
	\\
	& \quad \sum_{j \in V} x_{v_0j} \leq U \label{avrp:upper-k} \\
	& \quad \sum_{j \in V} x_{v_0j} \geq L \label{avrp:lower-k} \\
	\forall_{i, j \in V} & \quad x_{ij} \in \{0, 1\}
\end{align}

Another formulation is provided in \citet{Baldacci06}. This paper does not
assume, as the previous ones, that the trips to be serviced are predefined.
Furthermore, it addresses the problem of having multiple disposal facilities
and multiple inventory locations (where empty containers are stored). The
authors named this as the \textit{Multiple Rollon-Rolloff Vehicle Routing
Problem} (M-RRVRP). The M-RRVRP is converted to a \textit{Vehicle Routing
Problem with Time Windows} (VRPTW). Then, the problem is modelled as a
\textit{Set Partitioning} (SP) formulation.

A similar problem is defined by \citet{Archetti05}, named \textit{1-Skip
(container) Collection Problem} (1-SCP). Here, multiple disposal facilities are
available, and there are compatibility constraints between the facilities and
the containers. The trips that each vehicle can perform are different from the
four types defined in the RRVRP. Each vehicle starts its tour from a depot with
an empty container and travels to a location where there is a full one. The two
containers are then exchanged, and the vehicle proceeds with the full container
to a compatible disposal facility. It then proceeds to another trip, with a new
empty container. This model also considers time windows for picking up and
emptying containers.

The authors of \citet{Aringhieri04} also present a rather interesting
alternative for the Rollon-Rolloff model. They start by considering that
there is a finite number of available empty containers at the depot, $K_C$, and
compatibility constraints defined in the 1-SCP model are also present. Each
request for collection $i \in I, I = {1, ..., n}$ is characterized by its location
($\gamma_i$), container type ($\beta_{i}$) and waste material type ($\mu_i$). 

The graph model $G = (V, A)$ defines its vertices as representations of
collection requests. For each request $i$, there are two nodes $e_i$ and $f_i$
in $V$, that represent the full container to be collected and an empty
container to be delivered. As usual, vertex $v_0$ represents the
depot from where the vehicles must begin and end their tours. There are also
$K_C$ vertices (the set $D^\prime$), representing the possible pick up of an
empty container at the depot, and $K_C$ vertices (the set $D^{\prime\prime}$)
representing their delivery. No nodes will be added to represent the disposal
facilities; this information will be embed in the graph arcs.

Let $E = \{e_i : i \in I\}$, $F = \{f_i : i \in I\}$. Defining the set of
vertices as $V = \{v_0\} \cup E \cup F \cup D^\prime \cup D^{\prime\prime}$, we
now need to specify the arcs connecting them.

There is an arc between $f_i \in F$ and $e_j \in E$ if both services have the
same container type ($\beta_i = \beta_j$). These arcs correspond to picking up
a full container at $\gamma_i$, taking it to a disposal facility of type
$\mu_i$ for emptying and deploying it at location $\gamma_j$.

Between every $e_i \in E$ and $f \in F$ there is also an arc. It represents
deploying an empty container at location $\gamma_i$ and travelling to
$\gamma_j$ to pick up its full container.

At the start of a tour, a vehicle starts at the depot node, $v_0$, unloaded.
From here, there are two alternatives: it travels to a location with a full
container or it picks up an empty container. These operations are represented
by the arcs $(v_0, f), f \in F$ and $(v_0, d^\prime), \in D^\prime$.
Analogously, there are arcs that represent unloading an empty container and
finishing the tour. These arcs are defined by $(e, v_0), e \in E$ and
$(d^{\prime\prime}, v_0), d^{\prime\prime} \in D^{\prime\prime}$.

Picking up an empty container from the depot and deploying it is represented by
the arcs $(d^\prime, e), d^\prime \in D^\prime \wedge e \in E$. Picking up a
full container, emptying it at a disposal facility and dropping it at the depot
is modelled by the arcs $(f, d^{\prime\prime}), f \in F \wedge d^{\prime\prime}
\in D^{\prime\prime}$. Finally, there are cases where a vehicle drops an
empty container at the depot to pick up another empty one, of another type, or
a full one: $(d^{\prime\prime}, d^\prime), d^{\prime\prime} \in
D^{\prime\prime} \wedge d^\prime \in D^\prime$ and $(d^{\prime\prime}, f),
d^{\prime\prime} \in D^{\prime\prime} \wedge f \in F$.

To help clarify, an example graph is provided in figure~\ref{fig:avrp-rr}, with
two requests and an empty container at the depot, all with the same type.
Removing $v_0$, the graph should become bipartite, as a vehicle can only go
from nodes from which it leaves loaded to a node from which he leaves unloaded.

\begin{figure}[th]
  \begin{center}
    \leavevmode
    \begin{tikzpicture}[>=stealth']
\SetVertexMath
\Vertex[x=0,y=0.0]{v_0}
\Vertex[x=2,y=-2]{d''}
\Vertex[x=5,y=2.25]{e_0}
\Vertex[x=5,y=-2.25]{e_1}

\renewcommand*{\VertexLineWidth}{3pt}
\Vertex[x=2,y=2]{d'}
\Vertex[x=5,y=0.75]{f_0}
\Vertex[x=5,y=-0.75]{f_1}
\SetUpEdge[style=->]
\Edge(v_0)(d')
\Edges(d',e_0,d')
\Edges(d',e_1,d')
\Edge(d'')(v_0)
\Edges(f_0,e_0,f_0)
\Edges(f_1,e_1,f_1)
\Edge[style={<->,out=45,in=-45}](e_1)(f_0)
\Edge[style={<->,out=-45,in=45}](e_0)(f_1)
\Edge(v_0)(f_0)
\Edge(v_0)(f_1)
\Edge(f_0)(d'')
\Edge(f_1)(d'')
\Edge(e_0)(v_0)
\Edge(e_1)(v_0)
\Edge(d'')(d')

\end{tikzpicture}


    \caption{Graph for the Rollon-Rolloff problem. Thick vertices (except
    $v_0$) represent nodes from which the vehicle leaves loaded. Thin vertices
    are the ones from which it leaves unloaded.}
    \label{fig:avrp-rr}
  \end{center}
\end{figure}

This can now be modelled as an instance of AVRP. The costs $c_{ij}$ associated
with each arc are further detailed in \citet{Aringhieri04}.  $\delta^+(v)$ is
defined as the out-neighbourhood of $v$ and $\delta^-(v)$ as its
in-neighbourhood. Without the time constraints, the following formulation is
obtained:

\begin{align}
	\mbox{Minimize} & \quad K_A \sum_{(i, j) \in A} c_{ij} x_{ij} +
				K_B \sum_{j \in \delta^-(v_0)} x_{v_0j}
	\\
	\mbox{subject to} & \notag{}
	\\
	\forall_{i \in F \cup E} & \quad \sum_{j \in \delta^+(i)} x_{ij} = 1
	\label{avrp-rr:indegree}
	\\
	\forall_{i \in V} & \quad \sum_{j \in \delta^+(i)} x_{ij} = \sum_{j \in \delta^-(i)} x_{ji}
	\label{avrp-rr:in-out}
	\\
	& \quad \sum_{j \in V} x_{0j} \leq U \label{avrp-rr:upper-k} \\
	& \quad \sum_{j \in V} x_{0j} \geq L \label{avrp-rr:lower-k} \\
	\forall_{i \in D^\prime \cup D^{\prime\prime}} & \quad \sum_{j \in \delta^+(i)} x_{ij} \leq 1 \\
	\forall_{i, j \in V} & \quad x_{ij} \in \{0, 1\}
\end{align}

Although the graph is not complete, the missing arcs can be added with a
arbitrarily large value. If the restrictions described in
section~\ref{section:problem} need to be enforced, it is enough to limit the arcs on
the specified request. Say that the container from $f_0$ must return to $e_0$.
Removing the ingoing edges to $e_0$ and the outgoing edges from $f_0$ (except
the one that connects $f_0$ to $e_0$) is enough to ensure this constraint.




\subsection{Solving approaches}
\label{section:rrvrp-approaches}

Solving RRVRP instances, as modelled in the previous section, can be done using
the algorithms described in section~\ref{section:vrp-approaches}.



\section{Conclusion}

In the \textit{residential} scenario, there are several approaches applied to
the directed CARP. Although some have been adapted to the undirected variant,
further study could be made regarding the remaining algorithms --- namely
metaheuristics.

The \textit{Rollon-rolloff} scenario was formulated as a AVRP instance. Although
the authors provided preliminary computational results using an heuristic, further
computations could be made, comparing several other heuristics, metaheuristics and
exact methods.


