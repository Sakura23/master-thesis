\chapter{Problem description}
\label{chap:problem}

This chapter provides a description of the problem being addressed.


\section{Waste collection in Porto, Portugal}
\label{section:wcpp}

The municipal council of the city of Porto is interested in implementing a
platform that measures the fill status of the containers in real-time, so that
the evolution of waste can be monitored, the quality and efficiency of the
collection can be increased, and the sums paid to subcontractor companies by
amount of km traveled each month reduced. The system should work as follows:
containers send an alarm when they are full, and everyday in the evening the
waste collection routes are calculated with the static values available at a
certain time.

This project's scope is to build a framework that stores container fill status
and that uses that information to calculate efficient collection routes. Routes
have to be provided every day, based on recent fill information; although there
is no need to calculate routes in real-time, a solution must be provided within
1 to 2 hours. This constraint requires that an efficient solution is found
within that time frame for graphs of large cities, which are composed of
several thousand vertices. To decide which algorithm should be used to
calculate the routes, it was necessary to do a comparative study to evaluate
the performance of several techniques.

As stated in the previous chapter, a residential waste collection scenario is
modeled as a \textit{Capacitated Arc Routing Problem} (CARP), due to high number
of containers per street. CARP consists of, in summary, finding a set of vehicle
routes that visit a subset of a given graph's arcs. With each arc, there is an
associated demand, which must be serviced by exactly one vehicle. The sum of
the demands that a given vehicle services must not exceed a fixed limit.

Although residential waste collection scenarios are usually modeled as a CARP,
it must be taken into account that in Porto, containers service several homes
and that containers fill status will be monitored. With this in mind, it becomes
more natural to use a \textit{Asymmetric Capacitated Vehicle Routing Problem}
(ACVRP) model.  This model is similar to that of CARP, with a simple difference:
instead of servicing arcs, in ACVRP vehicles must service vertices instead.


\section{Asymmetric capacitated vehicle routing problem}
\label{section:problem-cvrp}

As exposed in the previous chapter, there are three non-exact methods to address
the \textit{Asymmetric Capacitated Vehicle Routing Problem}. One of them,
\textit{route-first-cluster-second}, first solves the associated
\textit{Asymmetric Traveling Salesman Problem} (ATSP); this implies finding the
shortest tour that visits every vertice exactly once. This process is followed
by the partitioning of the tour into routes that respect vehicle capacity
constraints (henceforth called clustering).

Another method, \textit{cluster-first-route-second}, first clusters the
containers and then attempts to find an optimum route for each cluster. This
method usually requires that one predefines the number of vehicles to use.

The third method to address the ACVRP is to use constructive heuristics, that
build all the vehicle routes in parallel.

This work will focus on the comparison of the first and third methods, using
different approaches to solve the ATSP. Additionally, special focus will be
given to the \textit{MAX-MIN ant system} (MMAS) --- which is described in
section~\ref{section:mmas} --- and its sensibility to certain parameters.

\section{Chapter summary}

This chapter summarized the motivation, goals and approach of this project. The
next chapter will focus on the definition of the architecure for the waste
sensor framework. It also presents the evaluation metrics and validation
datasets to analyse the optimization process.


