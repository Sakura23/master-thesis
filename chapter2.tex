\chapter{State of the art} \label{chap:sota}

\section*{}

In this chapter we will describe several studies regarding the optimization of
solid waste collection routes. We will also review the mathematical models used
in these studies and provide alternative optimization techniques that can be
applied to each one.

In section~\ref{section:problem} the route collection problem will be detailed.
Section~\ref{section:math} will mathematically formalize the three problem
variants. The following sections in this  chapter present approaches applied to
the cities of Trabon, Turkey\citep{Apaydin2007}, Barcelona,
Spain\citep{Bautista2004}, Athens, Greece\citep{Karadimas2005}, Hanoi,
Vietnam\citep{Tung2000} and Porto Alegre, Brazil\citep{Li2008}. 

\section{General problem description}
\label{section:problem}

As stated by \citet{Golden01}, one can divide waste collection routing problems
into three main categories. First, there is commercial collection, which
regards waste collection from businesses and organizations like malls,
factories and such. Second, the residential collection problem involves
collecting household generated waste, usually stored in containers along the
streets of a city.  Comparatively, the number of containers in the commercial
problem is significantly smaller than in the residential case.

In both of these two variants, each waste collection vehicle travels to a
container, loads its contents into its hopper and moves on to the next
container. As soon as the hopper is full, the vehicle travels to a landfill (or
other recycling/treatment facility) and deposits the collected waste.

The third variant --- named Rollon-Rolloff, which is described in
\cite{Bodin00} --- involves large containers, which must be transported
to the disposal facilities and replaced by empty ones. Their dimensions impose
that each vehicle can only transport a single container (either full or empty)
at a time. 

In the Rollon-Rolloff original description, each waste collection vehicle can
perform four basic operations:

\begin{itemize}
\item \textbf{Round trip} --- the vehicle picks up a container, brings it to a
waste disposal facility (WDF) for emptying and returns it to its original
location;

\item \textbf{Exchange trip} --- the vehicle leaves the WDF with an empty
container, travels to a location with a full one and switches them, bringing it
back to the WDF;

\item \textbf{Insertion trip} --- the vehicle brings an empty container from
the WDF to a new location;

\item \textbf{Removal trip} --- the vehicle picks up a full container from a
location and leaves it at the WDF.
\end{itemize}

In each of these three main variants, extra parameters can vary. For example,
we can consider a scenario with either a single or multiple waste disposal
facilities. In the multiple facilities case, an extra restriction may be to
balance the waste disposed at each location.

\subsection{Project focus}

Should this go here?

\section{Mathematical models}
\label{section:math}

All three variants described in section~\ref{section:problem} have a common
base for their mathematical models --- waste collection vehicles that travel
along the streets of a given city. To model a city, the common approach is to
define a graph in which each street is represented by one or two arcs
(depending if the street is one or both ways), with street intersections being
the vertices. Each arc may have several associated weights, reflecting the
street distance, the time it takes to transverse it, or some other metric. As
such, we define a city graph as the ordered tuple $G = (V, A)$, where $V$
represent the vertices and $A$ the arcs.

This definition alone does not suffice to realistically represent the possible
vehicle routes, as there are extra restrictions which must be considered,
specially regarding traffic signs. For example: at a given intersection, it may
not be possible to take a left turn if you are arriving from a specific street.
U-turns may also be forbidden, in certain intersections. 

These restrictions can be specified by defining a cost for every pair of arcs
that intersect at a given vertex.  We denote $t_{ava^\prime}$ as being the
cost to go from the arc $a$ to arc $a^\prime$ by making a turn at $v$. If the
arcs don't intersect at that vertex, or if the traffic signs forbid such a
turn, we set $t_{ava^\prime} = +\infty$. Further details on this approach can
be seen in \citet{Corberan2002887}.

For now, we shall consider the case in which there is a single disposal
facility, located in vertex $v_0 \in V$, and that all waste collection vehicles
start and finish their routes there. We consider that there are $K$ vehicles
available, and that each of the $K$ routes can be defined as a set of round
trips, each starting at the disposal facility. We may or may not decide to
limit the number of round trips in each route to one.

Further more, we consider vehicles being limited to an amount $Q$ of waste that
they can carry at any given time. This can be either measured in weight or in
volume.

\subsection{Residential scenario -- Capacitated arc routing problem}

In the residential scenario, it is usual to model it as a \textit{Capacitated Arc
Routing Problem} (CARP). Due to the container density per street, we consider
that vehicles should serve arcs instead of individual nodes.

A detailed description of CARP can be found in \citet{Sniezek01}, and a
mathematical model is provided in \citet{Belenguer98}. In the latter document,
it is assumed that each vehicle is limited to a single round trip. This happens
due to the chosen minimization criteria. Here, the goal was to minimize the
total cost of transversing and servicing the arcs in the calculated routes.
With this metric, allowing a vehicle to perform more than a single round trip
is the same as allowing for more than $K$ vehicles.

We consider that there is a subset $A_R \in A$ which represents the arcs that
must be serviced, and define $d_{a^\prime} \in \mathbb{R}$, with $a^\prime \in
A_R$, as the quantity of waste that needs to be collected at that arc. The
remaining arcs are called \textit{deadheading} arcs. The routes set is
identified by $I = \{1, ..., K\}$.

Then, we define a solution as a pair $(x, y)$, in which:

\begin{align*}
x_{ap} = & \quad \left\{
	\begin{array}{ll}
		1 & \quad \mbox{if vehicle $p \in I$ serves arc $a \in A_R$} \\
		0 & \quad \mbox{otherwise}
	\end{array}
	\right. \\
y_{ap} = & \quad \mbox{number of times vehicle $p \in I$ visits $a \in A$ without servicing it}
\end{align*}

From these variables, two extra functions are defined. The first,
$x_p(A_R^\prime) \in \mathbb{Z}_{\geq0}$, defines the number of arcs serviced
by vehicle $p \in I$ that are in $A_R^\prime \in A_R$. The second function,
$y_p(A^\prime) \in \mathbb{Z}_{\geq0}$, represents the total number of times
that vehicle $p \in I$ transverses the arcs in $A^\prime \in A$. These two functions can be defined as follow:

\begin{align}
	\forall_{p \in I, A_R^\prime \in A_R}	\quad x_p(A_R^\prime) = & \quad \sum_{a \in A_R^\prime} x_{ap} \\
	\forall_{p \in I, A^\prime \in A}	\quad y_p(A^\prime) =   & \quad \sum_{a \in A^\prime} y_{ap}
\end{align}

Another important definition is the \textit{cutset}. For a given set $S
\subseteq V$, $\delta(S) \in A$ represents the set of arcs that have one vertex
in $S$ and the other one in $V - S$. Additionally, we define $A(S) \in A$ as
the set of arcs with both endpoints in $S$. Their analogous functions,
$\delta_R(S)$ and $A_R(S)$ are restricted to the set of arcs which require
servicing.

\begin{align}
	\delta(S)	&= \{ (i, j) \in A : i \in S \wedge j \notin S \} \\
	A(S)		&= \{ (i, j) \in A : i,j \in S \} \\
	\delta_R(S)	&= \delta(S) \cap A_R \\
	A_R(S)		&= A(S) \cap A_R
\end{align}

With these definitions, \citet{Belenguer98} defines CARP as the following
Integer Programming:

\begin{align}
	\mbox{Minimize} & \quad \sum_{p \in I} \sum_{a \in A_R} c_a x_{ap} +
	\sum_{p \in I} \sum_{a \in A} c_{a} y_{ap}
	\\
	\mbox{subject to} &  \notag{} \\
	\forall_{a \in A_R} & \quad \sum_{p \in I} x_{ap} = 1
	\label{carp:required}
	\\
	\forall_{p \in I} & \quad \sum_{a \in A_R} d_a x_{ap} \leq Q
	\label{carp:capacity}
	\\
	\forall_{p \in I, S \subseteq V \setminus \{v_0\}, a \in A_R(S)} &
	\quad x_p(\delta_R(S)) + y_p(\delta(S)) \geq 2x_{ap}
	\label{carp:connectivity}
	\\
	\forall_{p \in I, S \subseteq V \setminus \{v_0\}} & \quad
	x_p(\delta_R(S)) + y_p(\delta(S)) = 0 \pmod{2}
	\label{carp:euler}
	\\
	\forall_{p \in I, a \in A} & \quad y_{ap} \in \mathbb{Z}_{\geq0} \\
	\forall_{p \in I, a \in A_R} & \quad x_{ap} \in \{0, 1\}
\end{align}

Constraints \eqref{carp:required} ensure that all arcs that required arcs are
served by one vehicle, while constraints \eqref{carp:capacity} ensure that each
vehicle does not collect more waste than its own capacity.

Constraints \eqref{carp:connectivity} imply that every vehicle $p$ that serves
an arc $a$ has its route connecting $v_0$ to $a$. Finally, constraints
\eqref{carp:euler} force each vehicle route to be a set of Eulerian circuits.

These constraints aren't strict enough so that every solution represents a set
of $K$ routes. There may be cases in which a vehicle route is defined by two or
more disconnected Eulerian circuits. This isn't considered as a problem,
because constraints~\eqref{carp:connectivity} force that all circuits
disconnected from $v_0$ contain only deadheading arcs. Since arc costs in the
minimization function are non-negative, each solution to the Integer Program
will has an associated K-route solution --- we just need to remove circuits
which are disconnected from $v_0$.

\subsection{Commercial scenario --- Capacitated vehicle routing problem}

When comparing the commercial and residential collection problems, one expects
that the number of containers is significantly lower in the first case. This
allows us to model the problem as a node/vertex routing problem, which is also
known as the \textit{Vehicle Routing Problem} (VRP). Applications of this model
are described in \citet{Tung2000} and \citet{Kim06}.

In this approach, each vertex $v \in V \setminus {v_0}$ has a demand $d_v \in
\mathbb{R}_{\geq0}$. Vehicle routes must cross all vertices whose demand is
greater than $0$. Although commercial waste containers aren't usually located
at street intersections, one can create additional vertices by splitting arcs.

Another technique would be to create $G^\prime = (V^\prime, A^\prime)$ in which
each vertex represents a waste container and the disposal facility. Each arc
$(i, j) \in A^\prime$ has an associated cost of travelling from vertex $i$ to
$j$, which can be easily obtained by applying a shortest path procedure to
graph $G$. Another consideration is that while $G^\prime$ has less vertices
than $G$, it is a fully connected graph. When using $G^\prime$ to define a
mathematical model, we consider that the vehicles must only visit each vertex
$v \in V^\prime \setminus \{v_0\}$ exactly once.

Since the graph is directed, this problem is usually named \textit{Asymmetric
Capacitated Vehicle Routing Problem} (ACVRP).  The most common mathematical
model\citep{Toth01} is based on the concept of $G^\prime$, and defines a
solution as a set of binary variables $x_{ij}$, one for each arc $(i, j)$.
Additionally, we consider $r(S), S \subseteq V \setminus \{v_0\}$ as the
minimum number of vehicles to serve all vertices in $S$. This value, that can
be determined by solving the \textit{Bin Packing Problem} (BPP), has a trivial lower
bound:

\begin{equation}
\forall_{S \subseteq V^\prime \setminus \{v_0\}} \quad r(S) \geq
\lceil\frac{1}{Q} \sum_{v \in V^\prime} d_{v} \rceil
\end{equation}

These definitions allows us to define ACVRP as the following Integer Programming:

\begin{align}
	\mbox{Minimize} & \sum_{i \in V^\prime} \sum_{j \in V} c_{ij} x_{ij}
	\\
	\mbox{subject to} & \notag{}
	\\
	\forall_{j \in V \setminus \{v_0\}} & \quad \sum_{i \in V} x_{ij} = 1
	\label{vrp:indegree}
	\\
	\forall_{i \in V \setminus \{v_0\}} & \quad \sum_{j \in V} x_{ij} = 1
	\label{vrp:outdegree}
	\\
	& \quad \sum_{i \in V} x_{i0} = K \label{vrp:indepot} \\
	& \quad \sum_{j \in V} x_{0j} = K \label{vrp:outdepot} \\
	\forall_{S \subseteq V \setminus \{v_0\}, S \neq \emptyset} & \quad
	\sum_{i \notin S} \sum_{i \in S} x_ij \geq r(S)
	\label{vrp:connectivity}
	\\
	\forall_{i, j \in V} & \quad x_{ij} \in \{0, 1\}
\end{align}

Constraints \eqref{vrp:indegree} and \eqref{vrp:outdegree} impose that each
vertex is visited exactly once, while constraints \eqref{vrp:indepot} and
\eqref{vrp:outdepot} force that exactly $K$ vehicles leave and return to the
disposal facility vertex. Finally, constraint \eqref{vrp:connectivity} ensures
that each subset of vertices is served by at least the minimum number of
vehicles required to fulfill their demands.


\subsection{Rollon-Rolloff vehicle routing problem}

In the \textit{Rollon-Rolloff Vehicle Routing Problem} (RRVRP), as in the
commercial problem, the number of containers is reduced. It has been researched
in several studies, with an approach based on a node routing mathematical
structure\citep{Bodin00,Aringhieri04,Meulemeester97}. Bodin et al. describe
this problem as being a combination of ACVRP --- due to round, removal and
insertion trips --- and BPP --- due to exchange trips.


\section{Approaches}

\subsection{Ant colony optimization}


\footnote{Exemplo de nota de rodapé.}

