\chapter{Introduction} \label{chap:intro}

\section*{}


\section{Context}

Municipal solid waste (MSW) production has been increasing in the last few
years, along with economic growth\citep{McCarthy94}. This has led to the need
--- and subsequent development --- of efficient waste management solutions.
Waste management involves not only the collection, but also the transportation,
recycling and disposal of generated waste.

One can find several case studies reporting the methodologies used by different
countries, as well as reports describing the MSW generation quantities and
patterns.

According to \citet{Bhat1996}, up to 85\% of some cities' waste management
budget spent goes to MSW collection and disposal. With this in mind, route
optimization can be considered as an important field of study regarding the
improvement of MSW management processes.

Many cities around the world have studied and applied several optimization
techniques to their collection scenarios, which can be very different from each
other. In Portugal, however, there seems to be little research regarding this
subject. \citet{Passaro200397} describes the waste management scenario in
Portugal from 1996 and 2002, a period in which several improvements were made
due to changes in the legislation. In 2006, \citet{Magrinho20061477} further
report the legislation trends and present some statistics on the average MSW
generation rate. Concerning collection routing, \citet{Teixeira04} describes a
study for optimizing the collection of urban recyclable waste in the
center-littoral region of Portugal.


\section{Project(?)}
Description of the project, with the containers sensors and such.

\section{Document structure}


Chapter~\ref{chap:sota} starts by describing the optimization of collection
routing mathematically, along with several techniques used to solve this
problem.

%Chapter~\ref{chap:proposed} provides details on the scenario being
%studied


